\section{The pressure tensor of a Hookean equation of state.}

It can be shown by reasons of covariance that the only dependence of $\rho$ is on $f^A$ and $H^{AB}$. Expanded to second order, and assuming isotropicity, one may write the energy per particle as
\begin{equation}
\label{eqn:eos}
\begin{array}{rcl}
\epsilon & = & \displaystyle{m_p + m_p\varepsilon(N) + p(N)\frac{H^{AB}G_{AB}-3}{N}} \\
 && + \frac{m}{8}\left[ 2r\left(H^{AB}G_{AB}-3\right)^2 + q\left(H^{AB}G_{BC} - \delta^{A}{}_C\right)\left(H^{CD}G_{DA} - \delta^{C}{}_A\right) \right]
 \end{array}
\end{equation}
so that it depends only on the invariants of the deformation tensor $H^{A}{}_{B}$; here, $m_p$ is the (average) particle rest mass. The first order trace term, proportional to $p(N)$ represents the pressure at zero strain, i.e. the prestressed case. According to Beig and Schmidt (2003) the coefficients $q$ and $r$ are related to the Lam\'e coefficients as $\lambda = m_pNq$ and $\mu = m_pNr$. According to Landau and Lifshitz ``Theory of Elasticity'' the bulk modulus is given by $K=\lambda + 2/3 \mu$. The coefficient $\mu$ is known as the shear modulus or the modulus of rigidity.

The Piaola-Kirchhoff stress-tensor corresponding to this energy term is given by
\begin{equation}
\label{eqn:pkTensor}
\tau_{AB} = \frac{p(N)G_{AB}}{N} +
\frac{m_p}{4}\left[2rG_{AB}\left(H^{CD}G_{CD}-3\right) + q\left(H^{CD}G_{AC}G_{BD} - G_{AB}\right)\right]
\end{equation}
or in a slightly more isotropic looking form
\begin{equation}
\label{eqn:pkTensorIso}
\tau_{AB} = \left(\frac{p(N)}{N} +
\frac{m_pr}{2}\left(H^{CD}G_{CD}-3\right)-\frac{m_pq}{4}\right)G_{AB} +
\frac{m_pq}{4}\left(H^{CD}G_{AC}G_{BD}\right)
\end{equation}
Due to the oddity of the second term in \eqref{eqn:pkTensor}, I will show it in more detail, define first the tensor functional $I$ as
\begin{equation}
I[H^{AB}] = \left(H^{CD}G_{DE} - \delta^{C}{}_E\right)\left(H^{EF}G_{FC} - \delta^{E}{}_C\right)
\end{equation}
we have
\begin{eqnarray}
\frac{\partial I}{\partial H^{AB}}\epsilon^{AB} & =& \left.\frac{dI\left[H^{AB}+\lambda\epsilon^{AB}\right]}{d\lambda}\right|_{\lambda = 0}, \\
 & = & \epsilon^{CD}G_{DE}\left(H^{EF}G_{FC} - \delta^{E}{}_C\right) + \epsilon^{EF}G_{FC}\left(H^{CD}G_{DE} - \delta^{C}{}_E\right), \\
 & = & \epsilon^{AB}G_{BC}\left(H^{CD}G_{DA} - \delta^{C}{}_A\right) + \epsilon^{AB}G_{BC}\left(H^{CD}G_{DA} - \delta^{C}{}_A\right), \\
 & = & 2\epsilon^{AB}G_{BC}\left(H^{CD}G_{DA} - \delta^{C}{}_A\right).
\end{eqnarray}
This also illustrates the general way of taking derivatives with respect to tensor without the use of coordinates. Writing $H=G_{AB}H^{AB}$ then
\begin{equation}
\tau_{AB} = \frac{p(N)G_{AB}}{N} +
\frac{m_p}{4}\left[2rG_{AB}\left(H-3\right) + q\left(H^{CD}G_{AC}G_{BD} - G_{AB}\right)\right]
\end{equation}
and we have the pressure tensor from the pullback
\begin{equation}
p_{ab} = p(N)G_{ab}\frac{n}{N} +
\frac{m_pn}{4}\left[2rG_{ab}\left(H-3\right) + q\left(H_{ab} - G_{ab}\right)\right]
\end{equation}
(note that $H_{ab} = H^{CD}G_{AC}G_{BD} f^A{}_af^B{}_b \neq h_{ab}$). In the case of zero strain $H^{AB} = G^{AB}$ implying $G_{ab} = H_{ab}\equiv h_{ab}$, the trace $H=3$, and $n=N$, so that
\begin{equation}
p_{ab} = p(n)h_{ab}
\end{equation}
which is the normal perfect fluid pressure tensor.

One also needs to compute the various derivatives of equation of state
quantities. The equation of state for a Hookean material is
\eqref{eqn:eos}
\begin{equation}
\begin{array}{rcl}
\epsilon & = & \displaystyle{m_p + m_p\varepsilon(N) + p(N)\frac{H^{AB}G_{AB}-3}{N}} \\
 && + \frac{m_p}{8}\left[ 2r\left(H^{AB}G_{AB}-3\right)^2 + q\left(H^{AB}G_{BC} - \delta^{A}{}_C\right)\left(H^{CD}G_{DA} - \delta^{C}{}_A\right) \right]
 \end{array} \tag{\ref{eqn:eos}}
\end{equation}
which leads to the second Piaola-Kirchhoff tensor \eqref{eqn:pkTensor}
\begin{equation}
\tau_{AB} = \frac{p(N)G_{AB}}{N} +
\frac{m_p}{4}\left[2rG_{AB}\left(H^{CD}G_{CD}-3\right) +
q\left(H^{CD}G_{AC}G_{BD} - G_{AB} \right) \right]
\tag{\ref{eqn:pkTensor}}.
\end{equation}
First, lets deal with the Piaola-Kirchhoff tensor. The one derivative is
easy
\begin{equation}
\frac{\partial \tau_{AB}}{\partial H^{BC}} = 
\frac{m_p}{4}\left(2rG_{AB} G_{CD} +
q G_{AC}G_{BD} \right)
\end{equation}
the next derivative, not so much
\begin{eqnarray}
\nonumber\frac{\partial \tau_{AB}}{\partial f^C} & = & \left( \frac{p'(N)}{N} -
\frac{p(N)}{N^2}\right)\frac{\partial N}{\partial X^C}G_{AB} +
\frac{p(N)}{N} G_{AB,C} +
\frac{m_p}{4}\left[2rG_{AB,C}\left(H^{DE}G_{DE}-3\right)\right. \\
&& \left. + 2rG_{AB}H^{DE}G_{DE,C} +
q\left(H^{DE}G_{AD,C}G_{BE} + H^{DE}G_{AD}G_{BE,C} - G_{AB,C} \right)
\right].
\end{eqnarray}
Finally, we just need the body derivative of the specific energy,
\begin{eqnarray}
\nonumber\frac{\partial \epsilon}{\partial f^A} = \frac{\partial \epsilon}{\partial X^A} & = &
m_p\varepsilon'(N)\frac{\partial N}{\partial X^A} + \left( \frac{p'(N)}{N} - \frac{p(N)}{N^2}\right)
\frac{\partial N}{\partial X^A}(H^{BC}G_{BC} - 3) +
\frac{p(N)}{N}H^{BC}G_{BC,A} \\
&& + \frac{m_p}{4}\left(2rH^{BC}G_{BC,A}(H^{DE}G_{DE}-3) +
qH^{BC}G_{CD,A}(H^{DE}G_{EB} - \delta^D{}_B)\right).
\end{eqnarray}
This is assuming that the coefficients $q$ and $r$ are constant...
Nothing really simplifies very much, in any case.