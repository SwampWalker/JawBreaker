\section{Stress-energy and the second Piaola-Kirchhoff tensor}
 When the action of the elastic matter is given as
\begin{equation}
\label{eqn:action}
S[f] = \int{\rho(f,df,g)\sqrt{-g}d^4x}
\end{equation}
and I will immediately point out that this action can be written to depend only on $H^{AB}$ and $f^A$ since any other combination of $g$ and $df$ would not be covariant. The stress-energy tensor is given as
\begin{equation}
T_{ab} = 2 \frac{\partial\rho}{\partial g^{ab}} - \rho g_{ab}.
\end{equation}
or
\begin{equation}
T_{ab} =  \rho u_au_b + p_{ab}.
\end{equation}
so long as the pressure tensor $p_{ab}$ is given by
\begin{equation}
\label{eqn:pressureDefinition}
p_{ab} = 2 \frac{\partial\rho}{\partial g^{ab}} - \rho h_{ab}.
\end{equation}
Defining the energy per particle $\epsilon$ (related to the specific internal energy when there is only one particle species as $\epsilon = m(1 + \varepsilon)$) as
\begin{equation}
\rho = n\epsilon
\end{equation}
the pressure tensor is expanded
\begin{equation}
p_{ab} = 2 n\frac{\partial\epsilon}{\partial g^{ab}} + 2 \epsilon\frac{\partial n}{\partial g^{ab}} - \rho h_{ab}.
\end{equation}
The volume form associated with the body metric can be used together with the particle density to produce a particle 3-form
\begin{equation}
N_{ABC} = N \sqrt{G}\epsilon_{ABC}
\end{equation}
which can be pulled back to get the particle 3-form on the spacetime
\begin{equation}
n_{abc} = f^{A}{}_{a}f^{B}{}_{b}f^{C}{}_{c}N_{ABC}.
\end{equation}
The square of the particle density three form gives the number density squared on spacetime
\begin{equation}
n^2 = \frac{n_{abc}n^{abc}}{3!} = \frac{n_{abc}n_{def}g^{ad}g^{be}g^{cf}}{3!} \equiv \frac{n_{abc}n_{def}h^{ad}h^{be}h^{cf}}{3!}
\end{equation}
which can be written as a determinant
\begin{equation}
\label{eqn:numberDensity}
n^2 =  \frac{N_{ABC}N_{DEF}H^{AD}H^{BE}H^{CF}}{3!} = N^2G\frac{\epsilon_{ABC}\epsilon_{DEF}H^{AD}H^{BE}H^{CF}}{3!} \equiv N^2G \det(H^{..}).
\end{equation}
The derivative of a determinant is given by the Jacobi formula, so
\begin{equation}
\frac{\partial n^2}{\partial H^{AB}} = N^2G\frac{\partial \det(H^{..})}{\partial H^{AB}} = N^2 G \det(H^{..}) H_{AB} = n^2 H_{AB}
\end{equation}
and $H_{AB}$ is the inverse of $H^{AB}$, that is $H^{AB}H_{BC} = \delta^A{}_C$. This relation can be pushed forward
\begin{equation}
\frac{\partial n^2}{\partial g^{ab}} = \frac{\partial n^2}{\partial H^{AB}} \frac{\partial H^{AB}}{\partial g^{ab}} = 2nf^A{}_af^{B}{}_b\frac{\partial n}{\partial H^{AB}} = n^2 h_{ab}
\end{equation}
so that
\begin{equation}
\frac{\partial n}{\partial g^{ab}} = \frac{nh_{ab}}{2}
\end{equation}
and the pressure tensor can finally be written in terms of the second Piaola-Kirchhoff tensor $\tau_{ab} = 2\partial\epsilon / \partial g^{ab}$
\begin{equation}
p_{ab} = n\tau_{ab}.
\end{equation}
One should note that because the energy per particle can only depend on the metric $g^{ab}$ through the push forward $H^{AB}$ (as shown in Beig and Schmidt)
\begin{equation}
\tau_{ab} = 2 \frac{\partial \epsilon}{\partial g^{ab}} = 2\frac{\partial \epsilon}{\partial H^{AB}}\frac{\partial H^{AB}}{\partial g^{ab}} = \tau_{AB}f^A{}_af^{B}{}_b,
\end{equation}
i.e. there is a natural relation between the spacetime tensor and the body tensor. Further, if one defines $H = \det(H_{..}) = 1/\det(H^{..})$ then $n=N\sqrt{G/H}$ and the spacetime number density is the body number density scaled by the ratio of volumes. Finally note that because of the form of $\tau_{ab}$ there is the identity $u^bp_{ab} = 0$.