\section{Spherically symmetric spacetime, spherically symmetric body
manifold}

I consider a body manifold in areal coordinates $(R,\Theta,\Phi)$, a
spherically symmetric body manifold then has a metric
\begin{equation}
dS^2 = G_{AB}dX^AdX^B = e^{2\Sigma(R)}dR^2 + R^2(d\Theta^2 +
\sin^2\Theta d\Phi^2)
\end{equation}
while the spacetime metric in areal coordinates $(t,r,\theta,\phi)$  can be written
\begin{equation}
ds^2 = g_{ab}dx^adx^b = -e^{2\lambda(r)}dt^2 + e^{2\sigma(r)}dr^2 + r^2(d\theta^2 +
\sin^2\theta d\phi^2).
\end{equation}
Without losing generality (there is a two parameter family of rotations that result in an equivalent equilibria), we can assume that the configuration is a
radial displacement only,
\begin{equation}
(f^R(t,r,\theta,\phi), f^\Theta(t,r,\theta,\phi),
f^\Phi(t,r,\theta,\phi)) = (\xi(r), \theta, \phi)
\end{equation}
giving a simple configuration gradient
\begin{equation}
f^A{}_a = \left\{
\begin{array}{lr}
\frac{d\xi(r)}{dr} & A=R,a=r, \\
1 & A=\Theta,a=\theta\ \mathrm{ or }\ A=\Phi,a=\phi, \\
0 & \mathrm{otherwise}.
\end{array}
\right.
\end{equation}
Because of the complication of the resulting formulae, we will write
subsequent equations as one parameter families of functionals 
(parameterized by $r$) of the unknown functions $\xi$,
$\xi'$, $\sigma$ and $\lambda$ (a prime will hereafter denote an
ordinary derivative with respect to $r$). Everything will depend in
general on the parameter $r$ so its dependence will be dropped from the
notation. In order for the elastic matter
to be at static, the four-velocity must be proportional to
the timelike Killing vector, so
\begin{equation}
u^t = e^{-\lambda(r)}
\end{equation}
and all other components vanish. The pressure tensor, written as a
matrix, is given by
\begin{equation}
p_{ab} = \left[
  \begin{array}{cccc}
    0 & 0 & 0 & 0 \\
    0 & n\tau_{RR}\xi'^2 & 0 & 0 \\
    0 & 0 & n\tau_{\Theta\Theta} & 0 \\
    0 & 0 & 0 & n\tau_{\Phi\Phi} 
  \end{array}
\right],
\end{equation}
the mixed pressure tensor is
\begin{equation}
p^a{}_b = \left[
  \begin{array}{cccc}
    0 & 0 & 0 & 0 \\
    0 & n\tau_{RR}\xi'^2e^{-2\sigma} & 0 & 0 \\
    0 & 0 & \frac{n\tau_{\Theta\Theta}}{r^2} & 0 \\
    0 & 0 & 0 & \frac{n\tau_{\Phi\Phi} }{r^2\sin^2\theta}  \end{array}
\right],
\end{equation}
while the contravariant version is
\begin{equation}
p^{ab} = \left[
  \begin{array}{cccc}
    0 & 0 & 0 & 0 \\
    0 & n\tau_{RR}\xi'^2 e^{-4\sigma}& 0 & 0 \\
    0 & 0 & \frac{n\tau_{\Theta\Theta}}{r^4} & 0 \\
    0 & 0 & 0 & \frac{n\tau_{\Phi\Phi} }{r^4\sin^4\theta}
  \end{array}
\right].
\end{equation}
The two independent Einstein equations are then given by
\begin{equation}
e^{-2\lambda}\left(G_{tt} - 8\pi T_{tt}\right) =
\frac{1+e^{-2\sigma}\left(2\sigma' - 1\right)}{r^2} +
8\pi\rho[\sigma,\xi,\xi']
= 0
\end{equation}
and
\begin{equation}
G_{rr} - 8\pi T_{rr} = \frac{1 + 2r\lambda' - e^{2\sigma}}{r^2} -
8\pi n\tau_{RR}[\sigma,\xi,\xi']\xi'^2 = 0,
\end{equation}
and the functional dependence of $\rho$ and $\tau_{RR}$ are yet to be
shown. With the identification (usual for the derivation of the TOV equations)
$\sigma(r) = -1/2\ln{\left(1-\frac{2m(r)}{r}\right)}$ the equations
become
\begin{equation}
m' = 4\pi r^2\rho[m,\xi,\xi']
\end{equation}
and
\begin{equation}
\lambda' = \frac{m + 4\pi n\tau_{RR}\xi'^2r^3 (1-\frac{2m}{r})}{r(r-2m)}
= \frac{m + 4\pi r^3 p^r{}_r}{r(r-2m)}
\end{equation}
which are extremely analogous to the TOV equations where one has $p^r{}_r
 = p_{rr}/g_{rr} = p(r)$.

The strain tensor takes the following form
\begin{equation}
H^{AB}[\sigma,\xi'] = g^{ab}f^A{}_af^B{}_b = \left[
\begin{array}{ccc}
e^{-2\sigma}\xi'^2 & 0 & 0\\
0 & \frac{1}{r^2} & 0 \\
0 & 0 & \frac{1}{r^2\sin^2\theta} 
\end{array}
\right]
\label{eq:sphericalStrainTensor}
\end{equation}
its (normalized to zero) trace is
\begin{equation}
H[\sigma,\xi,\xi']\equiv H^{AB}G_{AB} -3= e^{2(\Sigma(\xi(r))-\sigma(r))}\xi'^2(r) +
2\frac{\xi(r)^2}{r^2} - 3
\end{equation}
and its inverse is
\begin{equation}
H_{AB}[\sigma,\xi'] =  \left[
\begin{array}{ccc}
\displaystyle{\frac{e^{2\sigma}}{\xi'^2}} & 0 & 0\\
0 & r^2 & 0 \\
0 & 0 & r^2\sin^2\theta 
\end{array}
\right].
\end{equation}
The trace of the square is a little more complicated. First,
\begin{equation}
H^{AB}G_{BC} - \delta^A{}_C = \left[
\begin{array}{ccc}
e^{2(\Sigma(\xi(r))-\sigma(r))}\xi'^2(r) - 1 & 0 & 0 \\
0 & \frac{\xi(r)^2}{r^2} - 1& 0 \\
0 & 0 & \frac{\xi(r)^2}{r^2} - 1
\end{array}
\right]
\end{equation}
then
\begin{equation}
H_2[\sigma,\xi,\xi'] \equiv \mathrm{Tr}(\mathcal{H}-\mathcal{I})^2 \equiv
\left(H^{AB}G_{BC} - \delta^A{}_C\right)\left(H^{CB}G_{BA} -
\delta^C{}_A\right) = \left(e^{2(\Sigma(\xi(r))-\sigma(r))}\xi'^2(r) -
1\right)^2 + 2\left(\frac{\xi(r)^2}{r^2} - 1\right)^2.
\end{equation}
The number density in spacetime is calculated from \eqref{eqn:numberDensity} and \eqref{eq:sphericalStrainTensor}
\begin{equation}
n[\sigma,\xi,\xi'] =
N[\xi]\left(\frac{\xi}{r}\right)^2\xi'e^{\Sigma[\xi]-\sigma}.
\label{eq:numberDensity}
\end{equation}

The $r$-component of the conservation equations can be written
\begin{equation}
\left(\delta_r^c + u_ru^c\right) \left(\partial_a p^a{}_c +
\Gamma^a_{ad}p^d{}_c - \Gamma^d_{ac}p^a{}_d\right) + \rho u^a
\left(\partial_a u_r - \Gamma^c_{ar}u_c\right) = 0.
\end{equation}
but only the $t$-component of the four-velocity is non-zero, so
\begin{equation}
\partial_r p^r{}_r +
\Gamma^a_{ar}p^r{}_r - \Gamma^d_{ar}p^a{}_d + \rho
\Gamma^t_{tr}.
\end{equation}
The Christoffel symbols are
\begin{eqnarray}
\Gamma^t_{tr} & = & \lambda' \\
\Gamma^r_{rr} & = & \sigma' \\
\Gamma^\theta_{\theta r} & = \Gamma^\phi_{\phi r} = \frac{1}{r}
\end{eqnarray}
and the radial derivative of the determinant (the contraction of the Christoffel symbols) is
\begin{equation}
\Gamma^a_{ar} \equiv \frac{\partial_r \sqrt{-g}}{\sqrt{-g}} =
\frac{\partial_r e^{\lambda+\sigma}r^2\sin
\theta}{e^{\lambda+\sigma}r^2\sin \theta} = \frac{2}{r} + \sigma' + \lambda'.
\end{equation}
Now note that $\tau_{\phi\phi} = \tau_{\theta\theta}\sin^2\theta$ so
that $p^\phi{}_\phi = n\tau_{\theta\theta}/(r^2) = p^\theta{}_\theta$
\begin{equation}
(\partial_r p^r{}_r)[\sigma,\sigma',\xi,\xi',\xi''] + \lambda'
\left( p^r{}_r[\sigma,\xi,\xi'] + \rho[\sigma,\xi,\xi']\right)
+\frac{2}{r}\left(p^r{}_r[\sigma,\xi,\xi'] -
p^\theta{}_\theta[\sigma,\xi,\xi']\right) = 0.
\end{equation}
The TOV equation for hydrostatic equilibrium falls directly out of this
equation. There $p^a{}_b = ph^a{}_b = p\delta^a{}_b$ for $a,b\neq t$, and
in particular for hydro $p^r{}_r = p^\theta{}_\theta =p$.


In the interests of simplicity, I define a first order system by promoting the radial pressure to a first class variable. The constituent equation for the pressure then becomes a first order nonlinear differential equation for the configuration $\xi$. Collecting and naming the elastic functionals (while replacing $\sigma$ with its relation to $m$)
\begin{equation}
\mathcal{D}[p^r{}_r,m,\lambda',\xi,\xi'] \equiv \partial_r p^r{}_r + \lambda'
\left( p^r{}_r + \rho[m,\xi,\xi']\right)
+\frac{2}{r}\left(p^r{}_r - p^\theta{}_\theta[m,\xi,\xi']\right) = 0,
\label{eq:functionalhydro}
\end{equation}
\begin{equation}
\mathcal{E}[m,m',\xi,\xi'] \equiv m' - 4\pi r^2\rho[m,\xi,\xi'] = 0,
\label{eq:functionalrr}
\end{equation}
\begin{equation}
\mathcal{F}[p^r{}_r,m,\lambda'] \equiv \lambda' - \frac{m + 4\pi r^3 p^r{}_r}{r(r-2m)}
\label{eq:functionaltt}
\end{equation}
\begin{equation}
\mathcal{G}[p^r{}_r,m,\xi,\xi'] \equiv n[m,\xi,\xi']\tau_{RR}[m,\xi,\xi']\xi'^2\left(1 - \frac{2m}{r}\right) - p^r{}_r = 0.
\label{eq:functionalelastic}
\end{equation}
This set of equations governs radial dependence of the set of physical variables $(p^r{}_r,m,\lambda,\xi)$ which are respectively the radial pressure, the gravitational mass, the logarithm of the lapse and the configuration. Note that because of the notation used to refer to background components, $\xi$ could also be written as $R$.

\subsection{Hookean equation of state pressure tensor}
This subsection specialises the above equations to the equation of state chosen above. The energy per particle is \eqref{eqn:eos}
\begin{equation}
\nonumber\epsilon = \displaystyle{m_p + m_p\varepsilon(N) + P(N)\frac{H[\sigma,\xi,\xi']}{N}}
 +  \frac{\lambda_l(N)}{8N}H[\sigma,\xi,\xi']^2 + \frac{\mu(N)}{4N}H_2[\sigma,\xi,\xi']
\end{equation}
where $\lambda_l$ is the Lam\'e coefficient subscripted to distinguish it from the metric potential. The equation of state quantities in the background are denoted with capitals in a manner similar to other quantities; the pressure is denoted $P$ and the number density $N$, however, the background energy per particle is denoted as $\varepsilon$. The term $H$ and $H_2$ denotes the trace and the trace of the square of the generalised right Cauchy-Green tensor (minus the metric).

The Piaola-Kirchhoff stress tensor is \eqref{eqn:pkTensor}
\begin{equation}
\tau_{AB} = \left(\frac{P(N)}{N} + \frac{\lambda_l(N)}{4N}H\right)G_{AB} + \frac{\mu(N)}{2N}\left(H^{CD}G_{AC}G_{BD} - G_{AB}\right).
\end{equation}
It should be clear that the isotropic component of the pressure is therefor $P_I = P + \lambda_l/4H$. Recall the relation between the gravitational mass and the potential $\sigma(r) = -1/2\ln{\left(1-\frac{2m(r)}{r}\right)}$ and $\Sigma(\xi) = -1/2\ln{\left(1-\frac{2M(\xi)}{\xi}\right)}$, with this the Piaola-Kirchoff tensor can be written as (dropping functional dependence notation to save space)
\begin{equation}
N\tau_{AB} = \left[
  \begin{array}{ccc}
    \left(P_I + \frac{\mu}{2}\left(e^{2\Sigma-2\sigma}\xi'^2 - 1\right)\right)e^{2\Sigma}& 0 & 0\\
    0 & \left(P_I + \frac{\mu}{2}\left(\frac{\xi^2}{r^2} - 1\right)\right)\xi^2& 0 \\
    0 & 0 & \left(P_I + \frac{\mu}{2}\left(\frac{\xi^2}{r^2} - 1\right)\right)\xi^2\sin^2\theta
  \end{array}
\right].
\end{equation}
Choosing the notation $\chi \equiv n/N = \left(\frac{\xi}{r}\right)^2\xi'e^{\Sigma[\xi]-\sigma}$ for the volume contraction factor (ratio of the determinants of the spatial and body metric), the physical (mixed) pressure tensor can be written
\begin{equation}
p^a{}_b = \left[
  \begin{array}{cccc}
    0 & 0 & 0 & 0 \\
    0 & \chi\left(P_I + \frac{\mu}{2}\left(e^{2\Sigma-2\sigma}\xi'^2 - 1\right)\right)e^{2\Sigma-2\sigma}\xi'^2 & 0 & 0 \\
    0 & 0 & \chi\left(P_I + \frac{\mu}{2}\left(\frac{\xi^2}{r^2} - 1\right)\right)\frac{\xi^2}{r^2} & 0 \\
    0 & 0 & 0 & \chi\left(P_I + \frac{\mu}{2}\left(\frac{\xi^2}{r^2} - 1\right)\right)\frac{\xi^2}{r^2}  \end{array}
\right].
\end{equation}
Note that when $\xi<r$ the shear term in the tangential pressure becomes negative. In this case, the body has expanded and the elastic terms reduce the tangential pressure. This will occur for the radial pressure only when $G_RR/g_rr\xi'^2<1$.