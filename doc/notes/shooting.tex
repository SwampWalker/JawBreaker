\section{Finding the strained neutron star solutions: shooting}

A nice method in one dimension is the shooting method. A reference solution is evolved to give the background solution. The equation of state is modified via phase transition, and a new central pressure is selected as a guess. This solution is evolved until the conserved particle number interior to a given radius equals the conserved particle number in the interior of the reference solution. This gives the variables $m$, $r$ and $p^r{}_r$ at the boundary. The constituent equation for $p^r{}_r$ is inverted to compute $\xi'$, and this is used to compute $p^\theta{}_\theta$. If this equals $p^r{}_r$ then the shooting method has worked. Otherwise, a new guess pressure is selected. The comparison process is eased by noting that a solution is desired to the following,
\begin{equation}
p^r{}_r - p^\theta{}_\theta = n\tau_{RR}\xi'^2\left(1 - \frac{2m}{r}\right) - \frac{n\tau_{\Theta\Theta}}{r^2} = 0,
\end{equation}
\begin{equation}
p^r{}_r - p^\theta{}_\theta = \chi\left(P_I + \frac{\mu}{2}\left(\frac{1-\frac{2m}{r}}{1-\frac{2M}{\xi}}\xi'^2 - 1\right)\right)\frac{1-\frac{2m}{r}}{1-\frac{2M}{\xi}}\xi'^2 - \chi\left(P_I + \frac{\mu}{2} \left(\frac{\xi^2}{r^2} - 1\right) \right) \frac{\xi^2}{r^2} = 0.
\end{equation}
The solution has the structure
\begin{equation}
\chi\left(P_I + \frac{\mu}{2}(a - 1)\right)a = \chi\left(P_I + \frac{\mu}{2}(b - 1)\right)b.
\label{eq:rootStructure}
\end{equation}
The first solution to this polynomial is $a=b$; expanding and setting $\xi = R_c$
\begin{equation}
\frac{1-\frac{2m}{r_c}}{1-\frac{2M}{R_c}}\xi'^2 = \frac{R_c^2}{r_c^2}.
\label{eq:aEqualsb1}
\end{equation}
Rearranging this formula, a sufficient condition for the radial and tangential pressures to be equal is
\begin{equation}
\xi' = \sqrt{\frac{R_c^2}{r^2}\frac{1-\frac{2M}{R_c}}{1-\frac{2m}{r}}} = \sqrt{\frac{R_c^2-2M R_c}{r^2 - 2 m r}}.
\end{equation}


The shooting method is only convenient because one can use explicit time-stepping evolution algorithms, chiefly 4th order Runge-Kutta. In order to make use of this method, the constituent equation for the pressure \eqref{eq:functionalelastic} must be inverted using Newton-Raphson algorithm to compute $\xi'$. This can then be used to compute $p^\theta{}_\theta$ and the other derivatives.

\subsection{Core collapse to black hole}
In the case that the core collapses completely to form a black hole then the interior surface must be at equilibrium with an interior vacuum. Since,
\begin{equation}
p^\theta{}_\theta = \chi\left(P_I + \frac{\mu}{2} \left(\frac{\xi^2}{r^2} - 1\right) \right) \frac{\xi^2}{r^2} = 0,
\label{eq:tangentialPressureVanishes}
\end{equation}
\begin{equation}
p^r{}_r = \chi\left(P_I + \frac{\mu}{2}\left(\frac{1-\frac{2m}{r}}{1-\frac{2M}{\xi}}\xi'^2 - 1\right)\right)\frac{1-\frac{2m}{r}}{1-\frac{2M}{\xi}}\xi'^2 = 0
\label{eq:radialPressureVanishes}
\end{equation}
and both $r\ne 0$ and $\xi' \ne 0$ (both refer to singular situations), then
\begin{equation}
P + \frac{\lambda_l}{4}\left( \frac{1 - \frac{2m}{r}}{1 - \frac{2M(\xi)}{\xi}}\xi'^2 + 2\frac{\xi^2}{r^2} - 3 \right)+ \frac{\mu}{2} \left(\frac{\xi^2}{r^2} - 1\right) = 0,
\label{eq:tangentialBHCondition}
\end{equation}
\begin{equation}
P + \frac{\lambda_l}{4}\left( \frac{1 - \frac{2m}{r}}{1 - \frac{2M(\xi)}{\xi}}\xi'^2 + 2\frac{\xi^2}{r^2} - 3 \right) + \frac{\mu}{2}\left(\frac{1-\frac{2m}{r}}{1-\frac{2M}{\xi}}\xi'^2 - 1\right) = 0.
\label{eq:radialBHCondition}
\end{equation}
Rearranging the equations for $\xi'^2$,
\begin{equation}
P + \frac{\lambda_l}{4}\left(2\frac{\xi^2}{r^2} - 3 \right) + \frac{\mu}{2} \left(\frac{\xi^2}{r^2} - 1\right) + \frac{\lambda_l}{4} \left( \frac{1 - \frac{2m}{r}}{1 - \frac{2M(\xi)}{\xi}}\right)\xi'^2 = 0,
\label{eq:tangentialBHCondition2}
\end{equation}
\begin{equation}
P + \frac{\lambda_l}{4}\left(2\frac{\xi^2}{r^2} - 3 \right) - \frac{\mu}{2} + \frac{2\mu + \lambda_l}{4} \left( \frac{1 - \frac{2m}{r}}{1 - \frac{2M(\xi)}{\xi}}\right)\xi'^2 = 0.
\label{eq:radialBHCondition2}
\end{equation}
The two equations can then be combined to eliminate $\xi'^2$,
\begin{equation}
P + \frac{\lambda_l}{4}\left(2\frac{\xi^2}{r^2} - 3 \right) + \frac{\mu}{2} \left(\frac{\xi^2}{r^2} - 1\right) - \frac{\lambda_l}{2\mu + \lambda_l}\left( P + \frac{\lambda_l}{4}\left(2\frac{\xi^2}{r^2} - 3 \right) - \frac{\mu}{2} \right) = 0.
\label{eq:quadraticBHRadius}
\end{equation}
This yields a simple quadratic equation in $1/r^2$
\begin{equation}
\frac{\xi^2}{2}\left(\lambda_l + \mu - \frac{\lambda_l^2}{2\mu + \lambda_l}\right)\frac{1}{r^2} = \left(\frac{\lambda_l}{2\mu + \lambda_l} - 1\right)\left(P - 3\frac{\lambda_l}{4} - \frac{\mu}{2}\right)
\label{eq:quadraticBHRadius2}
\end{equation}
or
\begin{equation}
r = \sqrt{
\frac{\frac{\xi^2}{2}\left(\lambda_l + \mu - \frac{\lambda_l^2}{2\mu + \lambda_l}\right)}{\left(1 - \frac{\lambda_l}{2\mu + \lambda_l}\right)\left(3\frac{\lambda_l}{4} + \frac{\mu}{2} - P\right)}
}.
\label{eq:quadraticBHRadius2}
\end{equation}
Since $mu\ll\lambda_l$ the above equation is catastrophically inaccurate. Modifying it to make a more clear appoximation apparent,
\begin{equation}
r = \sqrt{
\frac{\xi^2\lambda_l\left(\frac{\mu}{2\mu + \lambda_l} + \frac{\mu}{\lambda_l}\right)}{2\frac{\mu}{2\mu + \lambda_l}\left(3\frac{\lambda_l}{4} + \frac{\mu}{2} - P\right)}
},
\label{eq:quadraticBHRadius3}
\end{equation}
\begin{equation}
r = \sqrt{\frac{\xi^2\lambda_l}{3\frac{\lambda_l}{2} + \mu - 2P}}\sqrt{1 + \frac{\mu}{\lambda_l}\frac{2\mu + \lambda_l}{\mu}},
\label{eq:quadraticBHRadius4}
\end{equation}
finally giving
\begin{equation}
r = \sqrt{\frac{4\xi^2\lambda_l}{3\lambda_l + 2\mu - 4P}}\sqrt{1 + \frac{\mu}{\lambda_l}}.
\label{eq:quadraticBHRadius5}
\end{equation}
Note that this only has a solution when $\lambda_l > 4/3P$, this inequality is generally (but not generically) true.